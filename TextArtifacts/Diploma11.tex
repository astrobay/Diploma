\documentclass[oneside,final,12pt]{extreport}
\usepackage[utf8]{inputenc}
\usepackage[russianb]{babel}
\usepackage{vmargin}
\setpapersize{A4}
\setmarginsrb{3cm}{2cm}{1.5cm}{2cm}{0pt}{0mm}{0pt}{13mm}
\usepackage{indentfirst}
\usepackage{graphicx}
\usepackage{setspace}
\usepackage{amsmath}
\onehalfspacing
\sloppy 
\rmfamily
\begin{document}
\begin{titlepage}

\begin{centering}
\vskip0.4cm
\includegraphics{GZ} \\
\vskip0.8cm
Московский государственный университет имени М.В. Ломоносова \\
Факультет вычислительной математики и кибернетики \\
Кафедра нелинейных динамических систем и процессов управления \\
\vskip2cm
\begin{Large} 
Куваев Даниил Юрьевич \\
\bigskip
{\bfseries Реализация освещения с использованием сферических гармоник} \\
\end{Large} 
\bigskip
ВЫПУСКНАЯ КВАЛИФИКАЦИОННАЯ  РАБОТА \\
\end{centering}
\vfill
\begin{flushright}
{\bfseries Научный руководитель:} \\
кандидат ф-м.н., мл. научный сотрудник \\
А.В. Боресков \\
\end{flushright}
\vfill
\centerline{Москва, 2019}

\end{titlepage}
\setcounter{page}{2}
\tableofcontents
\chapter{Введение}
\section{Понятийный аппарат}
Перед началом хотелось бы задать некий теоретический базис, без которого было бы невозможно продолжать дальнейшую работу. \\
{\bfseries PBR} \\
PBR, или физически-корректный рендеринг (physically-based rendering) это набор техник визуализации, в основе которых лежит теория, довольно хорошо согласующаяся с реальной теорией распространения света. \\
{\bfseries BRDF} \\
Для введения такого понятия, как BRDF, кратко рассмотрим вопрос взаимодействия света и материалов. \\
Предположим, что нам даны направления падающего и отражённого света 
\[ 
BRDF(w_i,w_0)=\frac{dL_0(w_0)}{dE_i(w_i)}=\frac{dL_0(w_0)}{L_i(w_i)\cos \Theta_i \, dw_i} 
\]
{\bfseries Задача реального времени} \\
{\bfseries Понятие ортонормирвоанной системы и сферических гармоник} \\

\section{Постановка задачи}
Обоснование выбора системы сферических гармоник \\
Формулировка \\ 
Пока хз что писать, но разложение бликовой компоненты с помощью сферических гармоник (для задачи реального времени!) и сравнение с классическим методом(?) \\
Актуальность  \\
Нам не удалось найти подобную реализацию, только для диффузной компоненты, искали на самых крутых конференциях \\

\chapter{Основная часть}
Обосновать выбор сферической системы, почему гармоники?(гуглить)
Теоремы про гармоники, почему они полны и т.д.
Понять, что такое L
Расписать разложение для диффузной компоненты
\[
L_0=\iint\limits_{\Omega_+} BRDF(w_i)\cos \Theta_i L(w_i)\, dw_i
\]
Можем выполнить переобозначение 
\(BRDF(w_i)\cos \Theta_i = K(w_i) \), 
так как информация о \(\Theta_i\) уже содержится в \( w_i\). Тогда, используя разложение функций по базису ортнормированной системы сферических гармоник, получим: 
\[ 
L_0=\iint\limits_{\Omega_+}K(w_i)L(w_i)\, dw_i = 	\langle K(w_i),L(w_i) \rangle \approx \langle \sum_{i=0}^{n} k_i\varphi_i(w_i) ,\sum_{j=0}^{n} l_j \varphi_j(w_i) \rangle \approx \sum_{i,j=0}^{n} k_i l_j \delta_i{}_j \approx \sum_{i=0}^{n} k_i l_i
\]
Здесь 
\(\delta_i{}_j = \langle \varphi_i(w_i),\varphi_j(w_i) \rangle =  \iint\limits_{\Omega_+} \varphi_i(w_i) \varphi_j(w_i)\, dw_i = 
\left \{\begin{aligned}
& 1 , i=j \\ & 0 , i \neq j
\end{aligned}\right.
\) — символ Кронекера, а коэффициенты \(k_i\) и \(l_j\) равны: \\
\[k_i = \iint\limits_{\Omega_+}K(w_i)\varphi_i(w_i)\, dw_i \] 
\[l_j = \iint\limits_{\Omega_+}L(w_i)\varphi_j(w_i)\, dw_i
\]
Расписать разложение для бликовой компоненты
Программная реализация для демонстрации? 
Сравнение сложности? Сравнение с качеством картинки при другом подходе? 



\end{document}
